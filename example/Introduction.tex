\documentclass{lablet}

\begin{document}

\title{Demo 0: Short introduction}

\maketitle

\begin{labletsheet}{Lab Activity Introduction}
	\lablettext{Experiments in Lablet can be presented in one of the following two forms:}
	\horizontaltwo
		{
		\labletheader{Lab Activities}
		\lablettext{- Each Lab Activity can have one ore more pages}
		\lablettext{- A page can have experiments, analysis, questions, graphs and other components}
		\lablettext{- You can create you own custom Lab Activities, e.g. to design an experiment for your classes}
		}
		{
		\labletheader{Single Experiments}
		\lablettext{- The sensors of the tablet can be used to record data}
		\lablettext{- The recorded data can then be analysed}
		}
	\labletheader{Even this introduction is a simple Lab Activity.}
	\lablettext{A Lab Activity can have multiple pages. Once all tasks on a page are solved, you can swipe to the next page or press the "Next" button.}
	\labletcheck{Check this box to unlock the next page}
\end{labletsheet}

\begin{labletsheet}{Thanks for trying Lablet!}
	\lablettext{- To get a better overview of Lablet's capabilities try the other Lab Activity demos}
	\lablettext{- To start a single experiment select "Single Experiments" from the action bar once you left this introduction}
\end{labletsheet}

\end{document}

