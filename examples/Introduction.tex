\documentclass{article}

\begin{document}

% >>lablet:title
\title{Demo 0: Short introduction}

\maketitle

% >>lablet:sheet
\section{Lab Activity Introduction}

    % >>lablet:text
	Experiments in Lablet can be presented in one of the following two forms:

	% >>lablet:horizontal

        % >>lablet:vertical
        % >>lablet:header
		\textbf{Lab Activities}

        % >>lablet:text
		- Each Lab Activity can have one ore more pages

        % >>lablet:text
		- A page can have experiments, analysis, questions, graphs and other components

        % >>lablet:text
		- You can create you own custom Lab Activities, e.g. to design an experiment for your classes

        % >>lablet:vertical
        % >>lablet:header
		\textbf{Single Experiments}

        % >>lablet:text
		- The sensors of the tablet can be used to record data

        % >>lablet:text
		- The recorded data can then be analysed

    % >>lablet:header
    \textbf{Even this introduction is a simple Lab Activity.}

    % >>lablet:text
    A Lab Activity can have multiple pages. Once all tasks on a page are solved, you can swipe to the next page or press the "Next" button.

    % >>lablet:check
	Check this box to unlock the next page

% >>lablet:sheet
\section{Thanks for trying Lablet!}
    % >>lablet:text
	- To get a better overview of Lablet's capabilities try the other Lab Activity demos

    % >>lablet:text
	- To start a single experiment select "Single Experiments" from the action bar once you left this introduction

\end{document}

