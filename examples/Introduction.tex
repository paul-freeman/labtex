\documentclass{article}

\begin{document}

% >>lablet:title{Demo 0: Short introduction}
% \title{Demo 0: Short introduction}

\maketitle

% >>lablet:beginsheet
\section{Lab Activity Introduction}
    % >>lablet:text
	Experiments in Lablet can be presented in one of the following two forms:
	% >>lablet:horizontal

		\textbf{Lab Activities}

		- Each Lab Activity can have one ore more pages

		- A page can have experiments, analysis, questions, graphs and other components

		- You can create you own custom Lab Activities, e.g. to design an experiment for your classes

		\textbf{Single Experiments}

		- The sensors of the tablet can be used to record data

		- The recorded data can then be analysed

	\textbf{Even this introduction is a simple Lab Activity.}

	A Lab Activity can have multiple pages. Once all tasks on a page are solved, you can swipe to the next page or press the "Next" button.

	Check this box to unlock the next page

\section{Thanks for trying Lablet!}
	- To get a better overview of Lablet's capabilities try the other Lab Activity demos

	- To start a single experiment select "Single Experiments" from the action bar once you left this introduction

\end{document}

