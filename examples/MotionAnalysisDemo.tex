\documentclass{article}

\begin{document}

% >>lablet:title
\title{Demo 2: Motion analysis}

\maketitle

% >>lablet:sheet
\section{Stage 1 Physics Laboratory}
    % >>lablet:header
	\textbf{Lab equipment:}
    % >>lablet:text
	Have you got the lab equipment? Check the following:
    % >>lablet:check
	a metre rule for setting the length scale in the video
    % >>lablet:check
	a ball

% >>lablet:sheet
\section{Take Videos:}
    % >>lablet:video{experimentFreeFall}
    Please take a free fall video:

    % >>lablet:video{experimentUpDown}
    Please take an up/down video:

    % >>lablet:video{experimentProjectile}
    Please take a projectile video:

% >>lablet:sheet
\section{Info}
	% >>lablet:text
	Check with your demonstrator about your videos and get ticked off before
	you proceed to video analysis.

	% >>lablet:text
	Video analysis: In this section, you and your lab partner will track the
    motion of the ball. For all three videos follow these steps:

	% >>lablet:text
	Go to "Video Settings". Set the start and end frames.  For example, the
	free-fall video should start shortly before the ball leaves your hand and
	shortly after the ball hits the ground. Optionally, increase the frame
	rate to 15 or 30 frames per second. Click "Apply".

	% >>lablet:text
	Now drag the green length scale to fit your length reference. Click on
    "Calibration" and set scale. Click "Apply".

	% >>lablet:text
	You can now tag the positions of the ball from the initial position to the
    final position. Find the cross-hair with green rings, drag it by the outer ring
    to tag the ball and advance to the next frame. Repeat until you have finished
    tagging. Click “Done” to analyse other videos (e.g. vertical linear motion and
    projectile motion).

% >>lablet:motionsheet{Mark Data Points}{experimentFreeFall}{Please analyse the free fall video:}

% >>lablet:motionsheet{Mark Data Points}{experimentUpDown}{Please analyse the up/down video:}

% >>lablet:motionsheet{Mark Data Points}{experimentProjectile}{Please analyse the projectile video:}

% >>lablet:calcysheet{Free fall}{experimentFreeFall}{Deriving average y-velocity and y-acceleration from displacement}

% -- SHEET 8
% >>lablet:sheet
\section{Free Fall}
	%>>lablet:header
	Estimating impact velocity
	% >>lablet:text
	Go to your demonstrator once you finished answering the questions on this page:
	    % >>lablet:motionanalysisgraph{experimentFreeFall}{showXVsYPosition}
	    % >>lablet:motionanalysisgraph{experimentFreeFall}{showTimeVsYSpeed}

	% >> lablet:question
	We know that moving objects have kinetic energy and that energy can be
transferred from one form to the other. How did the ball gain kinetic energy?

	% >> lablet:question
	Use the “Conservation of Energy” principle to estimate the impact velocity
of the ball. How does it compare with your observed final velocity? If there is
any difference, suggest the possible causes.
	% >> lablet:check
	Go to your demonstrator now and verify your results.

% -- SHEET 9
% >>lablet:sheet
\section{Vertical linear motion}
    % >>lablet:header
	Analysing graphs:

	% >> lablet:text
	Use the position-time graph and the velocity-time graph to complete the
questions below.

        % >>lablet:text
		<graph
			title="y-Position vs. Time"
			xAxis="time"
			yAxis="y-position"
			video="experimentUpDown">
		</graph>

		% >>lablet:text
		<graph
			title="y-Velocity vs. Time"
			xAxis="time\_v"
			yAxis="y-velocity"
			video="experimentUpDown">
		</graph>

    % >>lablet:text
	How does the vertical velocity vary with time?
	% >>lablet:text
	Point out on the height-time graph where the vertical velocity of the ball
was the maximum and also where it was zero.
    % >>lablet:header
	Estimating Energy Input
	% >>lablet:text
	Complete the following questions:
	% >>lablet:text
	<potentialEnergy1Question
		massText="Please enter the mass of the ball:"
		heightText="What was the height of the ball at its peak?"
		energyText="How much energy input enabled the ball to reach this height?"
		pbjText="A typical peanut butter jam (PBJ) sandwich contains 432 calories. How many throws could you perform with one PBJ sandwich? (1 calorie = 4.184 J)">
	</potentialEnergy1Question>
	% >>lablet:check
	Go to your demonstrator now and verify your results.

% >> lablet:sheet
\section{Projectile Motion}
    %>>lablet:header
	Analysing graphs
	% >>labelt:text
	The position-time graphs and the velocity-time graphs are created from your
measurements. Use these graphs to complete the questions below.
        % >>lablet:motionanalysisgraph{experimentProjectile}{showXVsYPosition}
        % >>lablet:motionanalysisgraph{experimentProjectile}{showTimeVsYSpeed}
	% >>lablet:text
	How does the vertical velocity vary with time?
	% >>lablet:text
	How would you draw a free body diagram of the ball at the peak of the
trajectory?
    % >>lablet:text
	Compare the “vertical velocity-time” graphs of projectile motion, vertical
linear motion and free fall. What are the similarities or the differences?
(Hint: How does the direction of the moving ball affect the sign of velocity?
Does the velocity vary or remain constant? Does the acceleration vary or remain
constant?
    % >> lablet:header
	Analysing the horizontal velocity in projectile motion
	% >>lablet:motionanalysisgraph{experimentProjectile}{showTimeVsXSpeed}
	% >>lablet:text
	Use the “horizontal velocity-time” graph to answer the following question:
	% >>lablet:text
	Does the horizontal velocity vary or remain constant?  What about the
horizontal acceleration?
    % >>lablet:header
	Estimating the horizontal and vertical accelerations
	% >>lablet:text
	Using the velocity-time graphs, estimate the horizontal and vertical
accelerations. How did you arrive to your estimates.
        % >>lablet:motionanalysisgraph{experimentProjectile}{showTimeVsXSpeed}
        % >>lablet:motionanalysisgraph{experimentProjectile}{showTimeVsYSpeed}
    % >>lablet:header
	Sources of uncertainties
	% >>lablet:text
	How do your estimates of horizontal and vertical accelerations compare with
your expected values? What are your expected values?  If there is any
difference, discuss the sources of uncertainties.
    % >>lablet:check
	Go to your demonstrator now and verify your results.
	% >>lablet:header
	Please export your data:
	% >>lablet:export

\end{document}

