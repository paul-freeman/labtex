\documentclass{lablet}

\begin{document}

\title{Demo 2: Motion analysis}

\maketitle

\begin{labletsheet}{Stage 1 Physics Laboratory}
	\labletheader{Lab equipment:}
	\lablettext{Have you got the lab equipment? Check the following:}
	\labletcheck{a metre rule for setting the length scale in the video}
	\labletcheck{a ball}
\end{labletsheet}

\begin{labletvideopage}{Take Videos:}
    \horizontal
    \labletvideo{experimentFreeFall}{Please take a free fall video:}
    \labletvideo{experimentUpDown}{Please take an up/down video:}
    \labletvideo{experimentProjectile}{Please take a projectile video:}
\end{labletvideopage}

\begin{labletsheet}{Info}
	\lablettext{ Check with your demonstrator about your videos and get ticked off before
                 your proceed to video analysis.}
	\lablettext{ Video analysis: In this section, you and your lab partner will track the
                 motion of the ball. For all three videos follow these steps:}
	\lablettext{ Go to "Video Settings". Set the start and end frames.  For example, the
                 free-fall video should start shortly before the ball leaves your hand and
                 shortly after the ball hits the ground. Optionally, increase the frame rate to
                 15 or 30 frames per second. Click "Apply".}
	\lablettext{ Now drag the green length scale to fit your length reference. Click on
                 "Calibration" and set scale. Click "Apply".}
	\lablettext{ You can now tag the positions of the ball from the initial position to the
                 final position. Find the cross-hair with green rings, drag it by the outer ring
                 to tag the ball and advance to the next frame. Repeat until you have finished
                 tagging. Click “Done” to analyse other videos (e.g. vertical linear motion and
                 projectile motion).}
\end{labletsheet}

\labletmotionpage{Mark Data Points}{experimentFreeFall}{Please analyse the free fall video:}

\labletmotionpage{Mark Data Points}{experimentUpDown}{Please analyse the up/down video:}

\labletmotionpage{Mark Data Points}{experimentProjectile}{Please analyse the projectile video:}

\labletcalcyspeed{Free fall}{experimentFreeFall}{Deriving average y-velocity and y-acceleration from displacement}

\begin{labletsheet}{Free Fall}
	\labletheader{Estimating impact velocity}
	\lablettext{Go to your demonstrator once you finished answering the questions on this page:}
	\horizontaltwo
		{\lablettext{<graph video="experimentFreeFall" type="X vs Y"></graph>}}
		{\lablettext{<graph video="experimentFreeFall" type="Time vs Y speed"></graph>}}
	\lablettext{We know that moving objects have kinetic energy and that energy can be
transferred from one form to the other. How did the ball gain kinetic energy?}
	\lablettext{Use the “Conservation of Energy” principle to estimate the impact velocity
of the ball. How does it compare with your observed final velocity? If there is
any difference, suggest the possible causes.}
	\labletcheck{Go to your demonstrator now and verify your results.}
\end{labletsheet}

\begin{labletsheet}{Vertical linear motion}
	\labletheader{Analysing graphs:}
	\lablettext{Use the position-time graph and the velocity-time graph to complete the
questions below.}
	\horizontaltwo
		{
		\lablettext{<graph
			title="y-Position vs. Time"
			xAxis="time"
			yAxis="y-position"
			video="experimentUpDown">
		</graph>}
		}
		{
		\lablettext{<graph
			title="y-Velocity vs. Time"
			xAxis="time\_v"
			yAxis="y-velocity"
			video="experimentUpDown">
		</graph>}
		}
	\lablettext{How does the vertical velocity vary with time?}
	\lablettext{Point out on the height-time graph where the vertical velocity of the ball
was the maximum and also where it was zero.}
	\labletheader{Estimating Energy Input}
	\lablettext{Complete the following questions:}
	\lablettext{<potentialEnergy1Question
		massText="Please enter the mass of the ball:"
		heightText="What was the height of the ball at its peak?"
		energyText="How much energy input enabled the ball to reach this height?"
		pbjText="A typical peanut butter jam (PBJ) sandwich contains 432 calories. How many throws could you perform with one PBJ sandwich?\\(1 calorie = 4.184 J)">
	</potentialEnergy1Question>}
	\labletcheck{Go to your demonstrator now and verify your results.}
\end{labletsheet}

\begin{labletsheet}{Projectile Motion}
	\labletheader{Analysing graphs}
	\lablettext{The position-time graphs and the velocity-time graphs are created from your
measurements. Use these graphs to complete the questions below.}
	\horizontaltwo
		{\lablettext{<graph video="experimentProjectile" type="X vs Y"></graph>}}
		{\lablettext{<graph video="experimentProjectile" type="Time vs Y speed"></graph>}}
	\lablettext{How does the vertical velocity vary with time?}
	\lablettext{How would you draw a free body diagram of the ball at the peak of the
trajectory?}
	\lablettext{Compare the “vertical velocity-time” graphs of projectile motion, vertical
linear motion and free fall. What are the similarities or the differences?
(Hint: How does the direction of the moving ball affect the sign of velocity?
Does the velocity vary or remain constant? Does the acceleration vary or remain
constant?}
	\labletheader{Analysing the horizontal velocity in projectile motion}
	<graph video="experimentProjectile" type="Time vs X speed"></graph>
	\lablettext{Use the “horizontal velocity-time” graph to answer the following question:}
	\lablettext{Does the horizontal velocity vary or remain constant?  What about the
horizontal acceleration?}
	\labletheader{Estimating the horizontal and vertical accelerations}
	\lablettext{Using the velocity-time graphs, estimate the horizontal and vertical
accelerations. How did you arrive to your estimates.}
	\horizontaltwo
		{\lablettext{<graph video="experimentProjectile" type="Time vs X speed"></graph>}}
		{\lablettext{<graph video="experimentProjectile" type="Time vs Y speed"></graph>}}
	\labletheader{Sources of uncertainties}
	\lablettext{How do your estimates of horizontal and vertical accelerations compare with
your expected values? What are your expected values?  If there is any
difference, discuss the sources of uncertainties.}
	\labletcheck{Go to your demonstrator now and verify your results.}
	\labletheader{Please export your data:}
	\lablettext{<export></export>}
\end{labletsheet}

\end{document}

