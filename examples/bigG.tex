\documentclass{article}
\usepackage{graphicx}
\usepackage{amsmath}
\usepackage{labs}
\usepackage{amsfonts}
\usepackage{amssymb}
\usepackage{hyperref}
\usepackage{import}
\usepackage{natbib}
\usepackage{enumitem}

% >>lablet:title
\def\expttitle{Estimation of the Gravitational Constant}
\def\exptnumber{313}
\markright{Experiment \exptnumber: \expttitle}

\hypersetup{
  colorlinks,%
  citecolor=blue,%
  filecolor=blue,%
  linkcolor=blue,%
  urlcolor=blue
}
\begin{document}

\maketitle

% >>lablet:sheet
\section*{Aims}

% >>lablet:text
The main aim of the experiment is to estimate the gravitational
constant $G$. Secondary aims are to study torques, torsion balances,
and damped harmonic oscillations. The experiment also provides an
exercise and discussion on systematic errors.

\textbf{Warning:} Do not disturb the setup until you have read this
handout. Mishandling of the apparatus can set it in motion, and if
this happens you will need to wait hours for the motion to damp out.

\bibliographystyle{plain}
\bibliography{refs}

\section*{Introduction}
Of the fundamental forces, the one of which we are most-directly aware
is that of gravity. The force between two spherical masses $m$ and $M$
whose centre-to-centre separation is $r$ is given by Newton's law of
Universal Gravitation:%
\begin{equation}
  {\bf F}=-G\frac{mM\hat{\bf r}}{r^{2}},%
  \label{eq:G}
\end{equation}
where $G$ is the ``gravitational constant,'' and the direction of
$\hat{\bf r}$ is defined in Figure~\ref{fig:mM}. Because gravity is by
far the weakest of the fundamental forces, our estimate of $G$ is one
of the universal constants with the lowest precision.

\begin{figure}[h]
  \centering
  \def\svgwidth{0.5\columnwidth}
  \import{figs/}{NewtonlawG.pdf_tex}
  \caption{Two masses, separated by a distance $r$, and the
    direction of the unit vector $\hat{\bf r}$.}
  \label{fig:mM}
\end{figure}

Following in the footprints of
\href{http://en.wikipedia.org/wiki/Henry_Cavendish}{Henry Cavendish},
we are going to estimate this constant $G$.  The experiment you are
about to perform is not very different from the way Cavendish did his
in 1798.\footnote{Cavendish estimated the density of the Earth; his
  measurements included all the ingredients to estimate $G$, but he
  never did. Poynting did so, almost 100 years later, in 1894.}
Interestingly, our current ``best'' estimates of $G$ are not that
different from the Cavendish' result, either!

We will use a torsion balance, that is the modern-day equivalent of
Cavendish' balance in 1798. It consists of a bar suspended by a very
thin wire and carrying a small lead sphere of mass $m$ at each end
(Figure~\ref{fig:balance}). To twist the balance an angle $\theta$ from
its position at rest, requires a torque $$\tau = \kappa \theta,$$
where $\kappa$ is the torque constant, defined by the rigidity and
dimensions of the wire.

\section*{Torque on the balance with $M$ in Position A}
\begin{figure}
  \centering
  \def\svgwidth{0.4\columnwidth}
  \import{figs/}{balance.pdf_tex}\hspace{2cm}
  \def\svgwidth{0.45\columnwidth}
  \import{figs/}{torque1.pdf_tex}
  \caption{Left: Torsion balance with masses $m$, and torque constant
    $\kappa$, under the influence of two masses $M$. Right: Top view
    of the balance.}
  \label{fig:balance}
\end{figure}
The presence of a lead sphere of mass $M$,
%are external to the enclosure and situated
as shown in Figure~\ref{fig:balance}, exerts a force ${\bf F}$ on $m$
as defined by equation~\ref{eq:G}.  With $m$ rotating around an axis
with a radius $d$, the force ${\bf F}$ results in a torque
\begin{equation}
  \boldsymbol{\tau} = {\bf d} \times {\bf F},
  \label{eq:tau}
\end{equation}
where $\times$ denoted the cross product, and the direction of the
vector $\boldsymbol{\tau}$ is according to the right-hand rule.

\begin{enumerate}
\item Use the right panel of Figure~\ref{fig:balance}, to show that
  the torque on the torsion balance due to one mass $M$ is $\left(
    \frac{GmM}{b_0^{2}}\cos\theta_{0}\right) d $.
\end{enumerate}

With the balance at rest in position A at an angle $\theta_0$ from
the reference position, the gravitational attraction between the large
and small spheres provides a torque that is exactly balanced by the
restoring torque from the suspension wire:
\begin{align}
  2\left(  \frac{GmM}{b_0^{2}}\cos\theta_{0}\right)  d - \kappa\theta_{0}=0,
  \label{eqn:basic}%
\end{align}
where $b_0$ is the distance between the spheres of mass $m$ and $M$ at
rest.

\begin{enumerate}[resume]
\item Show that for small angles $\theta_{0}$, $\cos\theta_{0}\approx
  1$, and that%
\begin{align}
  G = \frac{\theta_0b_0^2}{2mMd}\kappa
  \label{eqn:basic}%
\end{align}
\end{enumerate}
This means that if we can just estimate the torque constant $\kappa$,
$G$ can be expressed in known terms. The way to get $\kappa$
experimentally, is to oscillate the system in the next section.

\section*{Exciting damped harmonic oscillations: moving $M$ to Position B}
When the large spheres are rotated to position B, the suspended system
will be deflected from its initial angular position
$\theta=\theta_{0}$ and eventually reach a new rest position at the
same angle on the opposite side of the reference position. However, the
torsion balance will oscillate before equilibrium is restored. How
many oscillations and their amplitude depend on the torque constant
and friction in the wire and drag of the balance in the air, captured
jointly in friction constant $\beta$.

The differential equation governing the oscillatory motion of the
suspended system can be derived by applying Newton's second law of
motion for a rotational system. The torque provided by the masses $M$
is balanced by the rotational acceleration, a rotation velocity (with
friction $\beta$) and the torque provided by the wire with constant
$\kappa$:
% $${\bf F} = m {\bf a} $$
%where the angle $\theta$ as a function of time is
%described by:
%With reference to Fig. \ref{fig:three}(b), if 
\begin{equation}
  I\frac{\mathrm{d}^{2}\theta}{\mathrm{d}t^{2}}+ 
  \beta\frac{\mathrm{d}\theta}{\mathrm{d}t} + \kappa\theta =
  2\frac{GmM}{b^{2}}d,
  \label{eqn:de}
\end{equation}
where the moment of inertia $I$ \cite[Chapter 18
in][]{feynman2013feynman}. For the two discrete masses $m$, this is
\begin{equation}
  I =  \sum_i m_id_i^2 = 2md^2.
  \label{eq:I}
\end{equation}

For a bar of length $2d$ and $\rho$ as its mass per unit length, the
moment of inertia is:
  \begin{equation}
    I =  \int_{-d}^{d} \rho x^2 dx.
    \label{eq:Icont}
\end{equation}

\begin{enumerate}[resume]
\item  Show that the total moment of inertia of rod and masses $m$ is
  \begin{equation}
    I=2md^{2}+\frac{1}{12}m_{\mathrm{bar}}l_{\mathrm{bar}}^{2}=2md^{2}+\frac
    {1}{12}m_{\mathrm{bar}}\left(  2d\right)  ^{2}=2md^{2}\left(  1+\frac{1}%
      {6}\frac{m_{\mathrm{bar}}}{m}\right).
    \label{eqn:G6}%
  \end{equation}
\end{enumerate}

\subsection*{The time-dependent forcing term}
Now we are almost ready to solve the inhomogeneous differential
equation of second order, as presented by Equation~\ref{eqn:de}. The
left-hand side is uniformly expressed in terms of $\theta(t)$, but the
forcing term on the right-hand side is a function of mass separation
$b$, and this separation is a function of time, as well.

\begin{enumerate}[resume]
\item Use Figure~\ref{fig:balance} to show that $b(t)$ can be expressed
  as
  \[
  a-b(t)=d\sin(\theta(t)).
  \]
\item And use the small-angle approximation and a first order Taylor
  series approximation \cite[e.g., Section 3.1 in][]{SniederVanWijk15}
  to show that
  \begin{equation}
    \frac{1}{b(t)^{2}}\approx\frac{1}{a^{2}}\left(1+\frac{2d}{a}\theta(t)\right).
    \label{eqn:masssep}%
  \end{equation}
\end{enumerate}
Equation~(\ref{eqn:de}) then becomes:%
\begin{equation}
  I\frac{\mathrm{d}^{2}\theta}{\mathrm{d}t^{2}}+\beta\frac{\mathrm{d}\theta
  }{\mathrm{d}t}+K\theta=2\frac{GmMd}{a^{2}},
  \label{eqn:2de}
\end{equation}
where
\begin{equation}
  K=\kappa-\frac{4GmMd^{2}}{a^{3}}=
  \frac{2GmMd}{b_0^{2}\theta_{0}}-\frac{4GmMd^{2}}{a^{3}}.
  \label{eqn:Kdef}
\end{equation}
\begin{enumerate}[resume]
\item Rearrange equation~(\ref{eqn:Kdef}), so that%
  \begin{equation}
    G=\frac{b_0^{2}\theta_{0}}{2mMd}
    \left(\frac{a^{3}}{a^{3}-2b_0^{2}d\theta_{0}}\right)  K.
    \label{eqn:G1}%
  \end{equation}
\end{enumerate}
All terms except $K$ on the right-hand side are constants that we can
measure in the lab, and $K$ can be estimated indirectly by experiment
as equation~(\ref{eqn:2de}) is a standard second order differential
equation which has a solution of the form:%
\begin{equation}
  \theta=A\exp\left(  -\frac{\beta t}{2I}\right)  \cos\left(  \omega
    t+\phi\right)  +\frac{2GMmd}{Ka^{2}},
  \label{eqn:2desol}%
\end{equation}
where $A$ and $\phi$ are constants and the angular frequency is
\begin{equation}
  \omega=\sqrt{\frac{K}{I}-\frac{\beta^{2}}{4I^{2}}},%
\end{equation}
or%
\begin{equation}
  K=I\left(\omega^2+\frac{\beta^{2}}{4I^{2}}\right).
  \label{eqn:G2}%
\end{equation}

\begin{enumerate}[resume]
\item Show that expression~\ref{eqn:2desol} is a solution to the
  differential equation~\ref{eqn:2de}.
\end{enumerate}

\subsection*{The exponential decay of the oscillations}
\begin{figure}
  \centering
  \def\svgwidth{0.4\columnwidth}
  \import{figs/}{lightsource.pdf_tex}\hspace{2cm}
  \def\svgwidth{0.45\columnwidth}
  \import{figs/}{oscillations.pdf_tex}
  \caption{Left: Positions ($\pm s_0$) of the light beam at a
    distance $L$ from the torsion balance. The light source is
    reflected from the mirror on the torsion balance under influence
    of the large masses $M$ in Positions A and B. Right: Oscillations
    induced by moving the large masses $M$ from Position A to Position
    B.}
  \label{fig:light}
\end{figure}
The oscillatory motion of the suspended system is studied by observing
the source light reflected by the torsion balance onto the scale at
distance $L$ from the balance. The light spot moves from
$s=-s_{0}$ and oscillates about $s=+s_{0}$ (Left panel of
Figure~\ref{fig:light}).  If $L$ is large (about $2.5\mathrm{\,m}$ in
the laboratory set-up), $s$ may be treated as the arc subtended by the
angle of deflection of the light beam which is twice the deflection of
the suspended system (i.e. $2\theta$):
\begin{enumerate}[resume]
\item Use Figure~\ref{fig:light} to write:%
  \begin{equation}
    \theta_{0}=\frac{2s_{0}}{4L}=\frac{s_{0}}{2L},
    \label{eqn:G4}%
  \end{equation}
\end{enumerate}
so that
\begin{equation}
  G=\frac{b_0^{2}s_0}{4mMdL}
  \left(\frac{a^{3}}{a^{3}-b_0^{2}ds_0/L}\right)
  \left(\omega^2+\frac{\beta^{2}}{4I^{2}}\right)I.
  \label{eqn:G5}%
\end{equation}

Thus $s$ is proportional to $\theta$ and a plot of $s$ against $t$
(see the right panel of Figure~\ref{fig:light}), may be used to
measure the quantities necessary for the determination of the
gravitational constant $G$. From the plot of $s$ against $t$, we can
determine $\theta_{0}$ and $\beta/I$, the former by measuring the
initial and final position of the light spot ($-s_{0}$ and $s_{0}$),
the latter by fitting the period and decay of the oscillations.


\begin{enumerate}[resume]
\item Insert equation~(\ref{eqn:G6}), to get the following equation for
  the gravitational constant expressed in terms of measurable
  quantities:%
  \begin{equation}
    G=\frac{\omega^{2}b_0^{2}ds_{0}}{2ML}\left[ 
      \frac{a^{3}}{a^{3}-b_0^{2}ds_{0}/L}\right]  
  \left[1+\left(\alpha/\omega\right)^2\right]  
  \left[  1+\frac{1}{6}\frac{m_{\mathrm{bar}}}{m}\right],
  \label{eqn:finalG}%
\end{equation}
where the damping parameter $\alpha =\beta/(2I)$.
\end{enumerate}
Equation~(\ref{eqn:finalG}) has also been arranged so that the
principal and secondary effects of the oscillatory motion of the
suspended system are clear. All the factors in the square brackets
represent secondary effects and contribute less than 10\% to the value
of $G$. From left to right, they account for (a) the varying
gravitational torque, (b) effects of damping, and (c) consideration of
the bar in the moment of inertia of the suspended system. However, the
first term is not of the same form as the others, yet: 
\begin{enumerate}[resume]
\item Use a first-order Taylor expansion \cite[e.g., Section 3.1
  in][]{SniederVanWijk15} to show that $\left[
    \frac{a^{3}}{a^{3}-b_0^{2}ds_{0}/L}\right] \approx
  [1+b_0^{2}ds_{0}/(a^3L)]$, so that
  \begin{equation}
    G=\frac{\omega^{2}b_0^{2}ds_{0}}{2ML}\left[ 1+\frac{b_0^{2}ds_{0}}{a^3L}\right]  
    \left[1+\left(\alpha/\omega\right)^2\right]
    \left[  1+\frac{1}{6}\frac{m_{\mathrm{bar}}}{m}\right].
    \label{eqn:finalG2}%
  \end{equation}
\end{enumerate}

\section*{Correction for the effect of the ``distant'' sphere}
\begin{figure}
  \centering
  \def\svgwidth{0.5\columnwidth}
  \import{figs/}{distantsphere.pdf_tex}
  \caption{Forces $F$ and $F'$ on mass $m$ due to both large spheres.}
  \label{fig:five}
\end{figure}
In the theory given above, the gravitational force exerted by the more
distant of the two large spheres has not been taken into account. This
additional force $F'$ has a component $f$ exactly opposite to the
force $F$ due to the closer sphere (Figure~\ref{fig:five}).
Based on the geometry of Figure~\ref{fig:five}, we can express $f$ in
terms of $F$ as follows:%
\[
f=F'\cos\psi=\frac{GmM}{c^{2}}\left(  \frac{b_0+2d\sin\theta_{0}}%
  {c}\right)  %=\gamma F
\]
\begin{enumerate}[resume]
\item Show that $f = \gamma F$,  with $F=GmM/b_0^{2}$ and%
  \begin{equation}
    \gamma\approx\frac{b_0^{3}+2b_0^{2}d\theta_{0}}{\left(  b_0^{2}+4d^{2}+4b_0d\theta
        _{0}\right)  ^{3/2}}=\frac{b_0^{3}+b_0^{2}ds_{0}/L}{\left(  b_0^{2}+4d^{2}%
        +2b_0ds_{0}/L\right)  ^{3/2}}. 
    \label{eqn:gamma}%
  \end{equation}
\end{enumerate}

To take into account the force $f$, we should multiply
equation~(\ref{eqn:basic}) by $\left[ 1-\gamma\right]$. As a result,
equation~(\ref{eqn:finalG}) for the gravitational constant should be
divided by $\left[1-\gamma\right]$.

\begin{enumerate}[resume]
\item Use a first-order Taylor expansion to show that $\left[
    \frac{1}{1-\gamma}\right] \approx [1+\gamma]$, and that our
  estimate of the gravitational constant is now
\begin{equation}
  G=\frac{\omega^{2}b_0^{2}ds_{0}}{2ML}
  \left[ 1+\frac{b_0^{2}ds_{0}}{a^3L}\right]  
  \left[1+\left(\alpha/\omega\right)
    ^{2}\right]  
  \left[  1+\frac{1}{6}\frac{m_{\mathrm{bar}}}{m}\right][1+\gamma].
  \label{eqn:finalG3}%
\end{equation}
\end{enumerate}

Correcting this systematic error, should improve your estimate of $G$,
but even recent estimates of $G$ must still suffer from systematic
errors, as their quoted estimates plus uncertainties do not overlap
\cite{gillies1997newtonian,mohr2005codata,schlamminger2014fundamental}.

\section*{Experiment}

%\subsection*{Procedure}%

After you have answered all the previous questions:
\begin{enumerate}[resume]
\item  Measure $L$, the distance from the mirror to the screen. From the
measurements given for the apparatus, determine $a$ and $b$:

\[%
\begin{tabular}
[c]{lcc}%
\multicolumn{3}{c}{\textbf{Data for the apparatus}}\\
distance $a$ from the centre of the large mass M to the centre line of the small masses & $4.22 \pm 0.05$ & cm\\
Mass of large spheres ($M$) & $1.500\pm0.005$ & kg\\
Centre-to-centre distance between small spheres $\left(  2d\right)  $ &
$10.00\pm0.05$ & cm\\
Thickness of glass enclosure (between outer faces of glass plates) &
$2.94\pm0.02$ & cm\\
Ratio of mass of bar to mass of small sphere $\left(  m_{\mathrm{bar}%
  }/m\right)  $ & $0.06\pm0.01$ &
\end{tabular}
\]
Record ~5 minutes before moving M with lights on to later set the scale.

\item Turn on the light source for the experiment to reflect on the scale 
\item Measure distance from wall to scale
\item Set up the tablet with the tripod so that the light on the scale is in one end of the frame, and you have ~20 cm to the other side.
\item On the tablet, select the lablet application, then camera, video, and set frame rate to 0.2 fps
\item Start recording
\item After 5 minutes, flip the masses, turn off the lights and record for two more hours.
\item Stop recording (square button), select "done", and "save." This will result in a filename for your movie with the date in it.
\item Post-recording Analysis:
	a. Move scale bar, set length
	b. Set x/y coordinate system
\item Pick frames, use "view" button to troubleshoot bad picks with graph.
\item Connect tablet to a PC with USB, navigate to folder, select the csv file in your folder with the recording, and save this to the desktop PC
%\item Estimate G from the recordings and the equations in the handout.

% \item With the large spheres in the Position A (each sphere just
%   touching the side of the glass enclosure) and the system at rest
%   after having been left undisturbed for a considerable time, read the
%   position of the light spot. It is important that this reading be
%   made before the large spheres are moved. As this reading will be
%   used to determine $s_{0}$, several measurements should be taken over
%   about 10~minutes so that an average value can be obtained.

% \item Rotate the large spheres to Position B with the spheres
%   just touching the sides of the glass enclosure. Take readings of the
%   position of the light spot (e.g. every 15 seconds) and plot $s$
%   as a function of time.
  
\item Fitting the data to equation~(\ref{eqn:2desol}), determine the
  initial and final steady-state levels of $s$: $\pm s_{0}$, the
  angular frequency $\omega$, and the damping parameter $\alpha =\beta/(2I)$.

\item Estimate $G$ from equation~(\ref{eqn:finalG2}).

\item Estimate $G$ from equation~(\ref{eqn:finalG3}), which corrects for
  the attraction between $m$ and the distant $M$.
  
\item Carry out a systematic error analysis, including consideration
  of the approximations made in the formulae for determining the
  gravitational constant $G$.
\end{enumerate}%

\subsection*{List of Equipment}%

%TCIMACRO{\TeXButton{EndEnumProcedure}{\endenumprocedure}}%
%%BeginExpansion
%\endenumprocedure
%%EndExpansion

\begin{enumerate}
\item  Gravitation torsion balance Leybold Model 332 10
\item  Light source Leybold Model 450 60
\item 50-cm scale
\item A stopwatch
\end{enumerate}

\noindent A. Chisholm and Z.C. Tan, June 1992\\
Updated by Kasper van Wijk, 2014
\end{document}
