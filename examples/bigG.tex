\documentclass{beamer}
 
\usepackage[utf8]{inputenc}
\usepackage{lablet} 
 
\title{Experiment 313: Estimation of the Gravitational Constant}
\author{Dr. Kasper van Wijk}
\institute{University of Auckland}
\date{2016}
 
 
 
\begin{document}
 
\frame{\titlepage}
 
\begin{frame}
\frametitle{Aims}
The main aim of the experiment is to estimate the gravitational constant G.
Secondary aims are to study torques, torsion balances, and damped harmonic
oscillations. The experiment also provides an exercise and discussion on
systematic errors.

Warning: Do not disturb the setup until you have read this handout. Mishandling
of the apparatus can set it in motion, and if this happens you will need to
wait hours for the motion to damp out.
\end{frame}
 
\begin{frame}
\frametitle{References}
[1] Richard Phillips Feynman, Robert B Leighton, and Matthew Sands. The Feynman
Lectures on Physics, Desktop Edition Volume I, volume 1. Basic Books, 2013.

[2] George T Gillies. The Newtonian gravitational constant: recent measurements
and related studies.  Reports on Progress in Physics, 60(2):151, 1997.

[3] Peter J Mohr and Barry N Taylor. Codata recommended values of the
fundamental physical constants: 2002. Reviews of Modern Physics, 77(1):1, 2005.

[4] Stephan Schlamminger. Fundamental constants: A cool way to measure big G.
Nature, 510(7506):478– 480, 2014.

[5] Roelof Karel Snieder and Kasper van Wijk. A guided tour of mathematical
physics. Cambridge University Press, 3rd edition, 2015.
\end{frame}

\begin{frame}
\frametitle{Introduction}
Of the fundamental forces, the one of which we are most-directly aware is that
of gravity. The force between two spherical masses m and M whose
centre-to-centre separation is r is given by Newton’s law of Universal
Gravitation where G is the “gravitational constant,” and the direction of r̂ is
defined in Figure 1. Because gravity is by far the weakest of the fundamental
forces, our estimate of G is one of the universal constants with the lowest
precision.

Following in the footprints of Henry Cavendish, we are going to estimate this
constant G. The experiment you are about to perform is not very different from
the way Cavendish did his in 1798.1 Interestingly, our current “best” estimates
of G are not that different from the Cavendish’ result, either!

We will use a torsion balance, that is the modern-day equivalent of Cavendish’
balance in 1798. It consists of a bar suspended by a very thin wire and
carrying a small lead sphere of mass m at each end (Figure 2).  To twist the
balance an angle θ from its position at rest, requires a torque where κ is the
torque constant, defined by the rigidity and dimensions of the wire.
\end{frame}

\begin{frame}
\frametitle{Torque on the balance with M in Position A}
The presence of a lead sphere of mass M , as shown in Figure 2, exerts a force
F on m as defined by equation 1.  With m rotating around an axis with a radius
d, the force F results in a torque where × denoted the cross product, and the
direction of the vector τ is according to the right-hand rule.

\begin{check}
Use the right panel of Figure 2, to show that the torque on the torsion balance
due to one mass M is
\end{check}

With the balance at rest in position A at an angle θ0 from the reference
position, the gravitational attraction between the large and small spheres
provides a torque that is exactly balanced by the restoring torque from the
suspension wire where b0 is the distance between the spheres of mass m and M at
rest.

\begin{check}
Show that for small angles θ0 , cos θ0 ≈ 1, and that
\end{check}

Experiment 313: Estimation of the Gravitational Constant

313-3

This means that if we can just estimate the torque constant κ, G can be expressed in known terms. The
way to get κ experimentally, is to oscillate the system in the next section.

Exciting damped harmonic oscillations: moving M to Position B
When the large spheres are rotated to position B, the suspended system will be deflected from its initial
angular position θ = θ0 and eventually reach a new rest position at the same angle on the opposite side
of the reference position. However, the torsion balance will oscillate before equilibrium is restored. How
many oscillations and their amplitude depend on the torque constant and friction in the wire and drag of
the balance in the air, captured jointly in friction constant β.
The differential equation governing the oscillatory motion of the suspended system can be derived by applying
Newton’s second law of motion for a rotational system. The torque provided by the masses M is balanced
by the rotational acceleration, a rotation velocity (with friction β) and the torque provided by the wire with
constant κ:
GmM
dθ
d2 θ
+ κθ = 2 2 d,
(5)
I 2 +β
dt
dt
b
where the moment of inertia I [Chapter 18 in 1]. For the two discrete masses m, this is
X
I=
mi d2i = 2md2 .
(6)
i

For a bar of length 2d and ρ as its mass per unit length, the moment of inertia is:
Z

d

I=

ρx2 dx.

(7)

−d

3. Show that the total moment of inertia of rod and masses m is
I = 2md2 +



1
1
1 mbar
2
2
mbar lbar
= 2md2 + mbar (2d) = 2md2 1 +
.
12
12
6 m

(8)

The time-dependent forcing term
Now we are almost ready to solve the inhomogeneous differential equation of second order, as presented by
Equation 5. The left-hand side is uniformly expressed in terms of θ(t), but the forcing term on the right-hand
side is a function of mass separation b, and this separation is a function of time, as well.
4. Use Figure 2 to show that b(t) can be expressed as
a − b(t) = d sin(θ(t)).
5. And use the small-angle approximation and a first order Taylor series approximation [e.g., Section 3.1
in 5] to show that


1
2d
1
≈ 2 1 + θ(t) .
(9)
b(t)2
a
a
Equation (5) then becomes:
I

d2 θ
dθ
GmM d
+β
+ Kθ = 2
,
dt2
dt
a2

where
K =κ−

4GmM d2
2GmM d 4GmM d2
=
−
.
a3
b20 θ0
a3

(10)

(11)

Experiment 313: Estimation of the Gravitational Constant

313-4

a0

T
a2

s
−s0

0

scale

s0

s0
a3

light source
a1
L

Time (s)

4θ0
Position A
θ0

θ0

reference position

−s0

reference position
Pos. A

Position B

Position B

Figure 3: Left: Positions (±s0 ) of the light beam at a distance L from the torsion balance. The light source
is reflected from the mirror on the torsion balance under influence of the large masses M in Positions A and
B. Right: Oscillations induced by moving the large masses M from Position A to Position B.
6. Rearrange equation (11), so that
G=

b20 θ0
2mM d



a3
3
a − 2b20 dθ0


K.

(12)

All terms except K on the right-hand side are constants that we can measure in the lab, and K can be
estimated indirectly by experiment as equation (10) is a standard second order differential equation which
has a solution of the form:


βt
2GM md
θ = A exp −
cos (ωt + φ) +
,
(13)
2I
Ka2
where A and φ are constants and the angular frequency is
r
β2
K
− 2,
ω=
I
4I
or

β2
K=I ω + 2
4I


2

(14)


.

(15)

7. Show that expression 13 is a solution to the differential equation 10.

The exponential decay of the oscillations
The oscillatory motion of the suspended system is studied by observing the source light reflected by the
torsion balance onto the scale at distance L from the balance. The light spot moves from s = −s0 and
oscillates about s = +s0 (Left panel of Figure 3). If L is large (about 2.5 m in the laboratory set-up), s may
be treated as the arc subtended by the angle of deflection of the light beam which is twice the deflection of
the suspended system (i.e. 2θ):
8. Use Figure 3 to write:
θ0 =
so that
G=

b20 s0
4mM dL



s0
2s0
=
,
4L
2L

a3
3
a − b20 ds0 /L



β2
ω 2 + 2 I.
4I

(16)

(17)

Experiment 313: Estimation of the Gravitational Constant

313-5

M
F = GM m/b2
m

F 0 = GM m/c2
ψ
reference position
f = F cos(ψ)
c

Figure 4: Forces F and F 0 on mass m due to both large spheres.
Thus s is proportional to θ and a plot of s against t (see the right panel of Figure 3), may be used to measure
the quantities necessary for the determination of the gravitational constant G. From the plot of s against t,
we can determine θ0 and β/I, the former by measuring the initial and final position of the light spot (−s0
and s0 ), the latter by fitting the period and decay of the oscillations.
9. Insert equation (8), to get the following equation for the gravitational constant expressed in terms of
measurable quantities:

h

i
ω 2 b20 ds0
a3
1 mbar
2
1 + (α/ω)
1+
,
(18)
G=
2M L
a3 − b20 ds0 /L
6 m
where the damping parameter α = β/(2I).
Equation (18) has also been arranged so that the principal and secondary effects of the oscillatory motion
of the suspended system are clear. All the factors in the square brackets represent secondary effects and
contribute less than 10% to the value of G. From left to right, they account for (a) the varying gravitational
torque, (b) effects of damping, and (c) consideration of the bar in the moment of inertia of the suspended
system. However, the first term is not of the same form as the others, yet:
h
i
3
10. Use a first-order Taylor expansion [e.g., Section 3.1 in 5] to show that a3 −ba2 ds0 /L ≈ [1+b20 ds0 /(a3 L)],
0
so that

h


i
2 2
2
ω b0 ds0
b ds0
1 mbar
2
G=
1 + 03
1 + (α/ω)
1+
.
(19)
2M L
a L
6 m

Correction for the effect of the “distant” sphere
In the theory given above, the gravitational force exerted by the more distant of the two large spheres has
not been taken into account. This additional force F 0 has a component f exactly opposite to the force F
due to the closer sphere (Figure 4). Based on the geometry of Figure 4, we can express f in terms of F as
follows:


GmM b0 + 2d sin θ0
f = F 0 cos ψ =
c2
c

Experiment 313: Estimation of the Gravitational Constant

313-6

11. Show that f = γF , with F = GmM/b20 and
γ≈

b30 + 2b20 dθ0
3/2

(b20 + 4d2 + 4b0 dθ0 )

=

b30 + b20 ds0 /L
3/2

(b20 + 4d2 + 2b0 ds0 /L)

.

(20)

To take into account the force f , we should multiply equation (4) by [1 − γ]. As a result, equation (18) for
the gravitational constant should be divided by [1 − γ].
i
h
1
≈ [1 + γ], and that our estimate of the gravita12. Use a first-order Taylor expansion to show that 1−γ
tional constant is now



i
ω 2 b20 ds0
b2 ds0 h
1 mbar
2
G=
1 + 03
1 + (α/ω)
1+
[1 + γ].
(21)
2M L
a L
6 m
Correcting this systematic error, should improve your estimate of G, but even recent estimates of G must
still suffer from systematic errors, as their quoted estimates plus uncertainties do not overlap [2, 3, 4].

Experiment
After you have answered all the previous questions:
13. Measure L, the distance from the mirror to the screen. From the measurements given for the apparatus,
determine a and b:

Data for the apparatus
distance a from the centre of the large mass M to the centre line of the small masses
Mass of large spheres (M )
Centre-to-centre distance between small spheres (2d)
Thickness of glass enclosure (between outer faces of glass plates)
Ratio of mass of bar to mass of small sphere (mbar /m)

4.22 ± 0.05
1.500 ± 0.005
10.00 ± 0.05
2.94 ± 0.02
0.06 ± 0.01

Record 5 minutes before moving M with lights on to later set the scale.
14. Turn on the light source for the experiment to reflect on the scale
15. Measure distance from wall to scale
16. Set up the tablet with the tripod so that the light on the scale is in one end of the frame, and you have
20 cm to the other side.
17. On the tablet, select the lablet application, then camera, video, and set frame rate to 0.2 fps
18. Start recording
19. After 5 minutes, flip the masses, turn off the lights and record for two more hours.
20. Stop recording (square button), select ”done”, and ”save.” This will result in a filename for your movie
with the date in it.
21. Post-recording Analysis: a. Move scale bar, set length b. Set x/y coordinate system
22. Pick frames, use ”view” button to troubleshoot bad picks with graph.
23. Connect tablet to a PC with USB, navigate to folder, select the csv file in your folder with the recording,
and save this to the desktop PC
24. Fitting the data to equation (13), determine the initial and final steady-state levels of s: ±s0 , the
angular frequency ω, and the damping parameter α = β/(2I).

cm
kg
cm
cm

Experiment 313: Estimation of the Gravitational Constant

313-7

25. Estimate G from equation (19).
26. Estimate G from equation (21), which corrects for the attraction between m and the distant M .
27. Carry out a systematic error analysis, including consideration of the approximations made in the
formulae for determining the gravitational constant G.

List of Equipment
1. Gravitation torsion balance Leybold Model 332 10
2. Light source Leybold Model 450 60
3. 50-cm scale
4. A stopwatch
A. Chisholm and Z.C. Tan, June 1992
Updated by Kasper van Wijk, 2014


\end{document}

